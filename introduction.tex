\chapter{Introduction}
\label{chap:intro}
\OnehalfSpacing
\setlength{\parindent}{2em}
Etant actuellement en dernière année d'un diplôme d'ingénieur Informatique Industrielle et Instrumentation (3I) au sein de Polytech'Grenoble, 
j'ai effectué un stage de 6 mois dans le laboratoire TIMA (Techniques de l'Informatique et de la Microélectronique 
pour l'Architecture des systèmes intégrés) à Grenoble. 
Le stage s'est déroulé sous la direction de Alban BOURGE (Doctorant), Frédéric ROUSSEAU et Olivier MULLER. 
Mon sujet de stage était \og Mise en œuvre d'un environnement de validation pour la migration de tâches sur FPGA \fg.

Ce stage m'a permis d'effectuer un travail d'ingénierie ainsi que de découvrir le monde de la recherche
dans un laboratoire international. Je souhaitais faire ce stage au sein d'un laboratoire tel que TIMA dans l'équipe System Level Synthesis, pour obtenir une bonne compétence et expérience sur le développement d'un système microélectronique, notamment sur le développement d'architecture sur une plateforme reconfigurable de type FPGA (cf. Section \ref{sec:fpga}). 

L'objectif de ce stage étant de valider un mécanisme de sélection des points d'arrêts de l'exécution d'un traitement sur une plateforme reconfigurable ainsi que d'évaluer ses performances. 
Ce mécanisme est envisagé dans le cadre de la thèse de mon maître de stage, M. Alban Bourge, qui va être expliqué plus tard dans le chapitre \ref{chap:sujet}. 
Le but de mon stage est de faire un démonstrateur qui nous permet de réaliser des migrations de tâches et d'évaluer ses performances sur un FPGA.
Le mécanisme mentionné ci-dessus a été simulé avec succès, il nous reste donc la validation dans un FPGA. 

Dans le cadre de ce stage, j'ai eu pour mission de configurer une plateforme sur un FPGA permettant les fonctions suivantes :
\begin{itemize}
\item exécuter des IP,
\item arrêter l'execution d'un traitement,
\item récupérer le contexte,
\item transférer le contexte vers le FPGA destination,
\item reprendre de l'execution.
\end{itemize}

Dans ce rapport de stage, je présenterais le laboratoire TIMA ainsi que l'équipe SLS dans laquelle j'ai effectué mon stage. Ensuite, j'approfondirais le sujet de stage et je
présenterais la méthode de gestion et de déroulement du stage. J'expliquerais alors les principes de base technologiques mise en œuvre. Dans le chapitre suivant, je ferais une présentation des solutions
que j'ai proposées pour répondre aux besoins demandés. Je présenterais l'implémentation effectuée et les résultats.
Et enfin, je finirais avec une conclusion de ce rapport de stage.