\chapter{Description du Sujet et Gestion de Stage}
\label{chap:sujet}
\OnehalfSpacing

\section{Description du Sujet de Stage}
\label{sec:sujet}

Le laboratoire TIMA développe un mécanisme de sélection des points
d'arrêts de l'exécution d'un traitement sur un FPGA. L'idée sous-jacente
est de pouvoir arrêter une tâche sur un FPGA et de transférer la suite de
l'exécution sur un autre FPGA.
Ce mécanisme est connu sur le système à base de CPU 
en tant que \emph{context switch} ou "commutation de contexte".
Un contexte sur FPGA est définie comme une image de l'état des circuits configurés dans le FPGA.

La commutation de contexte en informatique consiste à sauvegarder
l'état d'un processus pour restaurer à la place d'un autre dans le cadre de l'ordonnancement 
d'un système d'exploitation. La commutation de contexte sur FPGA (cf. Section \ref{sec:contextswitch}) 
effectue un traitement similaire sur des circuits logiques au lieu d'un système d'exploitation. 
Elle est possible grâce aux blocs logiques programmables
qui construisent la plateforme FPGA.

Dans le cadre de sa thèse au laboratoire TIMA, M. Alban Bourge a proposé une méthode de la commutation 
de contexte sur FPGA par l'extraction de l'image de l'état sur certains points de contrôle\cite{Bourge2015}. 
Un algorithme a été ajouté à chaque IP\footnote{\emph{Intelectual Property}, dans ce contexte est le circuit programmé sur FPGA} synthétisé
afin de choisir le meilleur point de contrôle selon les contraintes donnés par l'utilisateur.
Lors de la synthèse logique, l'IP est modifié afin qu'il puisse effectuer la commutation de contexte.

La réalisation de cette méthode a été faite par une méthode intégrée dans un outil de synthèse de haut niveau.
Ceci est réalisable grâce à un outil de synthèse logique de haut-niveau AUGH\cite{Prost2014} 
qui a été développé par un chercheur postdoctoral au laboratoire TIMA.
Le AUGH est capable de générer des IP 
depuis un programme écrit en langage C. L'intégration de la méthode proposée par M. Alban Bourge sur AUGH 
nous permet de générer un IP qui supporte les récupérations et restaurations de contexte sur FPGA.

Pendant le projet 3i5, une plateforme a été construite pour valider des IP générés par AUGH automatiquement\cite{Brisebard2015}.
Cette plateforme de validation consiste à un environnement de pilotage et une carte ZYBO\cite{zyboweb}
connecté à un serveur qui contient un outil de back-end Xilinx. L'environnement de pilotage permet d'envoyer les
vecteurs d'entrées, de recevoir les sorties et de les comparer avec la référence.
Les interfaces entrées-sorties personnalisées sont ajoutées dans l'architecture pour transférer des données entrées-sorties.
Cette plateforme est assez puissante pour construire un architecture à partir d'un IP généré par AUGH
avec les interfaces automatiquement.

En adaptant d'idée de la validation automatique du projet 3i5 au stage, une plateforme pour valider et évaluer les performances du mécanisme
de la commutation de contexte est proposée. La plateforme est composé d'une ZYBO connecté au réseau et d'un environnement de pilotage.
Comme le mécanisme de la commutation de contexte est intégré dans l'IP généré par AUGH,
le principe de génération d'architecture est similaire et les interfaces entrées-sorties peuvent être réutilisées avec quelques 
modifications. L'environnement de pilotage permet de
synthétiser les IP, de commander l'arrêt de l'exécution d'un IP, de récupérer du contexte, de transférer du contexte et
de reprendre l'exécution.

\section{Gestion de Stage}
\label{sec:gestion}

Afin d'augmenter l'efficacité du travail et d'atteindre le but de stage, un planning a été fait.
Ce planning est constitué en plusieurs parties.

La première partie est ....
Le premier mois a été consacré afin d'effectuer .....

La deuxième partie est 
Cette partie a été réalisé pendant .....

La troisième partie est

La semaine restante étant prévu pour l'écriture du rapport

Le déroulement du stage est présenté par la diagramme de gantt sur la Figure \ref{fig:ganttchart}.

\begin{figure}[h]
	\label{fig:ganttchart}
	\centering
	\includegraphics[width=1.7\textwidth,angle=90]{gantt}
	\caption{La diagramme de gantt de déroulement du stage}
	\vspace{-2mm}
\end{figure}

Les relations entre les tâches et les étapes sont montrés par la diagramme des tâches (Méthode PERT) sur la Figure \ref{fig:pert}.

\begin{figure}[h]
	\label{fig:pert}
	\centering
	\includegraphics[width=\textwidth, page=2]{stage}
	\caption{La diagramme des tâches (méthode PERT) du stage}
	\vspace{-2mm}
\end{figure}

