\chapter{Description du sujet et gestion de stage}
\label{chap:sujet}
\OnehalfSpacing
\section{Version simplifiée de la description du sujet de stage}

La migration de tâche est un terme courant dans le cadre d'ordonnancement 
de système d'exploitation multi-tâche. Elle permet de transférer une tâche
d'un processeur à un autre. Dans le traitement sur un FPGA, la migration d'une tâche
n'est pas encore bien développé car la configuration de matériel ne permet pas
d'arrêter une exécution au milieu et de la continuer sur un autre FPGA. Le composant
de traitement programmé sur un FPGA n'est pas capable de s'arrêter au milieu d'exécution.

M. Alban Bourge a proposé un mécanisme qui permet de la migration de tâche sur
FPGA en ajoutant du matériel au composant de traitement. 
En cours d'exécution, le traitement peut s'arrêter et les informations nécessaires
à la reprise d'exécution sont récupérées. Ce mécanisme sélectionne des points d'arrêts
dans le traitement qui permet de récupérer et de restaurer les informations.

Le sujet de stage concerne à la validation du mécanisme de sélection des points d'arrêts d'un traitement
sur le FPGA. Un démonstrateur est construit afin de valider ce mécanisme proposé et d'évaluer
ses performances.

\section{Description du sujet de stage}
\label{sec:sujet}

Le laboratoire TIMA développe un mécanisme de sélection des points
d'arrêts de l'exécution d'un traitement sur un FPGA. L'idée sous-jacente
est de pouvoir arrêter une tâche en cours d'exécution sur un FPGA et de transférer la suite de
l'exécution sur un autre FPGA. Ce mécanisme est connu sur le système à base de CPU 
en tant que «commutation de contexte ou \emph{context switch}».
Un contexte sur FPGA est définie comme une copie des informations de l'état du circuit configuré dans le FPGA.

La commutation de contexte en informatique consiste à sauvegarder
l'état d'un processus pour en mettre en place un autre dans le cadre de l'ordonnancement 
d'un système d'exploitation. La commutation de contexte sur FPGA (cf. Section \ref{sec:contextswitch}) 
effectue un traitement similaire sur une tâche matérielle (un circuit logique s'exécutant sur un FPGA) 
au lieu d'une tâche logicielle (un morceau de code assembleur s'exécutant sur un processeur)

Dans le cadre de sa thèse au laboratoire TIMA, M. Alban Bourge a proposé un mécanisme de la commutation 
de contexte sur FPGA par l'extraction de contexte sur certains points de contrôle\cite{Bourge2015} ou \emph{checkpoints}. 
Du matériel est ajouté à chaque composant de traitement ou \emph{Intelectual Property} (\gls{IP}) pour qu'il soit 
commutable au meilleur \emph{checkpoint} selon les contraintes données par l'utilisateur.

La réalisation de ce mécanisme est faite durant la génération du composant de traitement.
Lorsque le composant de traitement est généré à partir d'un programme par l'outil
de synthèse haut niveau (\emph{High Level Synthesis} ou \gls{HLS}), le matériel de 
la commutation de contexte est intégré. Un programme est ajouté à l'outil de HLS
afin de générer ce composant de traitement commutable.
Ceci est réalisable grâce à un outil de HLS AUGH\cite{Prost2014}
qui a été développé au laboratoire TIMA.
AUGH est capable de générer des composants de traitement 
depuis un programme écrit en langage C. L'intégration de ce mécanisme de \emph{checkpoints} dans AUGH 
nous permet de générer un composant de traitement qui supporte 
les récupérations et restaurations de contexte sur FPGA.

Afin de valider le mécanisme des \emph{checkpoints} et d'évaluer ses performances, un démonstrateur
est proposé dans ce stage. Ce démonstrateur doit être capable d'exécuter un traitement, de réaliser la commutation
de contexte aux \emph{checkpoints} de ce traitement et d'évaluer ses performances. Alors que le composant de traitement commutable
est programmé dans le démonstrateur, la gestion d'exécution du traitement est nécessaire à la réalisation.
Une plateforme qui est constituée d'une partie programmable (FPGA) et d'une partie contrôle (processeur)
est choisie pour l'implémentation.

La conception de l'architecture du démonstrateur est réalisée en adaptant d'une plateforme similaire construite auparavant.
Cette plateforme a été construite pendant le projet 3i5. Elle a été réalisée pour
valider des composants de traitement générés par AUGH automatiquement\cite{Brisebard2015, Wicaksana2015}.
Le composant de traitement est programmé par un processeur dans un FPGA. Le flot d'exécution, la gestion
d'entrées-sorties et l'évaluation des performances du composant de traitement sont fait par le processeur.
Une rédaction d'un article de ce travail du projet a été faite pendant le stage 
et il est soumis (mais refusé) à un workshop à Londres (cf. Annexe 1).

\section{Gestion de stage}
\label{sec:gestion}
\justify
Dans le cadre de déroulement du stage, un planning a été fait.
Ce planning est modélisé sous la forme d'un diagramme de Gantt dans la Figure \ref{fig:gantt}. Pour une meilleure planification du projet, 
la méthode \gls{PERT} a été utilisée. La liste de tâche sous la forme d'un diagramme de PERT est présentée
dans la Figure \ref{fig:pert}.

Dans la Figure \ref{fig:gantt}, le stage est divisé en quatre grandes parties. 
La première partie est le développement matériel. Cette partie concerne 
l'intégration du composant de traitement et les composants annexes. La deuxième partie est le développement logiciel. Ce développement
concerne le système d'exploitation, les bibliothèques de logiciels qui seront embarqués et les scripts pour lancer le flot d'exécution.
La troisième partie de stage est le test. Dans tous les développements de système, les tests sont nécessaires
pour valider les composants (test unitaire) et l'intégration de tous les composants (test d'intégration). La dernière partie
est la documentation de stage. Cette documentation inclut la rédaction du rapport et la préparation à la soutenance.

Les deux premières semaines ont été consacrées à la modification des interfaces E/S personnalisées. Ces interfaces
gèrent les entrées et les sorties du composant de traitement généré par AUGH. Pour valider la commutation de contexte,
il faudrait s'assurer que l'application fonctionne correctement incluant les communications d'entrées-sorties.
Dans le projet 3i5, des interfaces similaires sont utilisées. Il suffit de les améliorer pour l'implémentation
de ce stage qui sera expliquée dans le Chapitre \ref{chap:implementation}.

Après la modification des interfaces E/S, l'étape suivante est le développement de l'interface de la commutation
de contexte qui a été effectué pendant environ un mois. Afin d'évaluer les performances du système, l'utilisateur doit pouvoir connaitre le
nombre des cycles d'exécution. Un timer est donc développé pendant deux semaines pour enregistrer
le nombre des cycles de l'exécution et de la commutation de contexte. 

Alors que la partie matérielle est développée pendant
8 semaines, la partie logicielle est développée pendant environ 6 semaines, incluant les bibliothèques de logiciels et l'application sur FPGA.
Les bibliothèques de logiciels sont de fonctions supplémentaires ajoutées dans la plateforme. L'application sur FPGA
constitue deux parties, la génération de configuration à installer sur FPGA et application à exécuter pour la validation des
migrations de tâches.

Dans le stage, la phase de tests est effectuée en même temps que les développements matériel et logiciel ainsi que les intégrations.
Le test unitaire est réalisé en faisant les simulations fonctionnelles pour le développement de la partie matérielle.
Le test de la partie logicielle est effectué par les compilations avec \gls{GCC} avant l'implémentation sur la plateforme.
Après que tous les composants soient intégrés,
le test global/intégration est effectué en même temps que le développement de l'application pour FPGA. 

Au milieu du stage, un article sur le projet 3i5\cite{Wicaksana2015} est rédigé pendant 10 jours. Cet article est soumis
à «Second International Workshop on FPGAs for Software Programmers (FSP 2015)», 
mais il a été refusé car sa présentation était hors sujet pour le workshop.
Le mois et demi restant est prévu pour le rédaction du rapport superposant le test d'intégration et la préparation
de la soutenance.

\begin{figure}[h]
	\centering
	\includegraphics[width=0.9\textheight,angle=90]{stage2}
	\caption{La diagramme de Gantt de déroulement du stage}
	\label{fig:gantt}
	\vspace{-2mm}
\end{figure}

Les dépendances entre les tâches sont présentées dans la Figure \ref{fig:pert}. 
Les 4 grandes parties du diagramme de Gantt sont présentées dans la figure par les 4 branches du graphe.
Les dates de fin «plus tôt» et «plus tard» sont mises dans chaque noyau. Le chemin critique du diagramme PERT
dans la figure est Début $\rightarrow$ Test Unitaire $\rightarrow$ Test Global $\rightarrow$ Fin. Ce sont les tâches les plus importantes et les plus longues car
elles valident les travaux effectués dans le stage. 
Le test unitaire est réalisé pendant le développement matériel et logiciel. Le test global est fait pour l'intégration
du système et développement de l'environnement pilotage.

\begin{figure}[h]
	\centering
	\includegraphics[width=0.9\textheight, angle=90]{pert}
	\caption{La diagramme des tâches (méthode PERT) du stage}
	\label{fig:pert}
	\vspace{-2mm}
\end{figure}

