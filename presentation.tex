\chapter{Présentation du laboratoire}
\label{chap:presentation}
\OnehalfSpacing

\section{Histoire et présentation}
TIMA est un laboratoire public de recherche sous la tutelle du Centre National de la Recherche Scientifique 
(CNRS), de l'Institut Polytechnique de Grenoble (Grenoble INP), et de l'Université Joseph Fourier (UJF). 
TIMA est une équipe cosmopolite, avec des chercheurs et stagiaires du monde entier. Une grande partie de la recherche s'effectue dans le contexte de projets coopératifs, avec des partenaires industriels et académiques, financés par des contrats régionaux, nationaux et européens.

Avec environ 120 personnes dont 64 doctorants (en moyenne sur les 4 dernières années), le laboratoire TIMA est situé au 46 Avenue Félix Viallet, Grenoble. Chaque année, TIMA obtient des contrats pour un total de 200.000 \euro\  environ par an, actuellement il y a 24 projets en cours.
\begin{figure}[h]
	\label{fig:maptima}
	\centering
	\includegraphics[width=0.6\textwidth]{maptima}
	\caption{Plan du Laboratoire TIMA}
	\vspace{-5mm}
\end{figure}

\justify
TIMA est à l'origine de 8 créations d'entreprises. Parmi les plus récentes, TIEMPO a été créée en 2007 pour commercialiser la technologie de conception de circuits asynchrones inventée par le Groupe CIS; UROMEMS est issue des résultats de thèse de Hamid Lamraoui (groupe MNS) en collaboration avec TIMC et l'hôpital de la Pitié Salpétrière à Paris. La société vise à résoudre les problèmes d'incontinence urinaire avec un nouveau concept de sphincter urinaire artificiel automatisé.

Les sujets de recherche du Laboratoire TIMA couvrent la spécification, la conception, la vérification, le test, les outils CAO et méthodes d'aide à la conception pour les systèmes intégrés, depuis les composants de base analogiques et numériques jusqu'aux systèmes multiprocesseurs sur la puce et à leur système d'exploitation de base.

Les membres de TIMA contribuent à de nombreuses manifestations internationales, participent à des actions de recherche européennes variées, et ont établi des accords scientifiques avec des universités et des laboratoires de recherche dans le monde entier.

\section{Recherches au TIMA}
Le laboratoire est organisé en 5 équipes de recherche :

\begin{enumerate}
\item\ AMfoRS : Architectures and Methods for Resilient Systems

L'équipe AMfoRS aborde les défis cruciaux dans le domaine des systèmes intégrés résilients.
Ses activités visent à augmenter les synergies entre les technologies de vérification 
et les activités de conception et de validation de systèmes intégrés fiables, sûrs et sécurisés, l'accent étant mis sur les parties numériques.

\item

CDSI : Design of Integrated devices, Circuits and Systems

L'axe Conception de Dispositifs, Circuits et Systèmes (CDSI) oeuvre dans le domaine de la conception, 
la fabrication et le test de dispositifs, circuits et systèmes intégrés. 
Son activité principale se concentre sur la minimisation de l'énergie et la miniaturisation de composants et systèmes.

\item
	
RIS : Robust Integrated Systems

L'équipe RIS aborde les défis fondamentaux induits par la miniaturisation nanométrique poussée, 
incluant: les densités de défauts très élevées causées par les variabilités 
du processus de fabrication des tensions d’alimentation et des températures, le vieillissement accéléré des circuits, 
les interférences électromagnétiques (IEM), et les soft errors ; ainsi que les contraintes de faible consommation.

\item

RMS : Reliable Mixed-signal Systems

L'équipe RMS a une expertise reconnu dans le domaine des systèmes et circuits intégrés analogiques/mixtes (AMS) 
et radiofréquence (RF), avec un accent spécifique sur les techniques de test / de contrôle embarqués et les outils de CAO associés.
Les techniques qu'ils développent visent à réduire les coûts de test de production, 
à augmenter le rendement de production, à améliorer la qualité, la fiabilité et la robustesse.

\item

SLS : System Level Synthesis (cf. Section \ref{sec:sls})

\end{enumerate}

\section{Groupe SLS}
\label{sec:sls}
Les défis que posent l'intégration de systèmes complets en technologies CMOS nanométrique sont fortement liés au nombre phénoménal de processeurs que l'on peut implanter sur une puce. L'ITRS prévoit plus de 1000 processeurs en 2020 pour les applications destinées au grand public. Dans ce contexte, le groupe SLS a identifié les axes suivants comme étant majeurs :
\begin{itemize}
\item\ la définition d'architectures parallèles, configurables et reconfigurables; qui permettent d'exploiter pleinement les possibilitées offertes par ces technologies ;
\item\ la partie logicielle de ces systèmes; qui est de plus en plus importantes, la définition d'infrastructures permettant de déployer de manière efficace des applications à grande échelle sur des systèmes contraints en ressources est une piste majeure ;
\item\ l'intégration matérielle/logicielle; étant d'une grande complexité, des outils de synthèse, de génération et de simulation performant et permettant un passage à l'échelle sont nécessaires.
\end{itemize}