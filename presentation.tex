\chapter{Présentation du Laboratoire}

\section{Histoire et Présentation}
TIMA est un laboratoire public de recherche sous la tutelle du Centre National de la Recherche Scientifique 
(CNRS), de l'Institut Polytechnique de Grenoble (Grenoble INP), et de l'Université Joseph Fourier (UJF). 
TIMA est une équipe cosmopolite, avec des chercheurs et stagiaires du monde entier. Une grande partie de la recherche s'effectue dans le contexte de projets coopératifs, avec des partenaires industriels et académiques, financés par des contrats régionaux, nationaux et européens.

Avec environ 120 personnes dont 64 doctorants (en moyenne sur les 4 dernières années), le laboratoire TIMA est situé au 46 Avenue Félix Viallet, Grenoble. Chaque année, TIMA obtient des contrats de 200.000 \euro\  environ par an, actuellement il y a 24 projets en cours.
\begin{figure}
	\label{fig:maptima}
	\centering
	\includegraphics[width=0.8\textwidth]{maptima}
	\caption{Plan du Laboratoire TIMA}
\end{figure}

\justify
TIMA est à l'origine de 8 créations d'entreprises. Parmi les plus récentes, TIEMPO a été créée en 2007 pour commercialiser la technologie de conception de circuits asynchrones inventée par le Groupe CIS; UROMEMS est issue des résultats de thèse de Hamid Lamraoui (groupe MNS) en collaboration avec TIMC et l'hôpital de la Pitié Salpétrière à Paris. La société vise à résoudre les problèmes d'incontinence urinaire avec un nouveau concept de sphincter urinaire artificiel automatisé.

Les sujets de recherche du Laboratoire TIMA couvrent la spécification, la conception, la vérification, le test, les outils CAO et méthodes d'aide à la conception pour les systèmes intégrés, depuis les composants de base analogiques et numériques jusqu'aux systèmes multiprocesseurs sur la puce et à leur système d'exploitation de base.

Les membres de TIMA contribuent à de nombreuses manifestations internationales, participent à des actions de recherche européennes variées, et ont établi des accords scientifiques avec des universités et des laboratoires de recherche dans le monde entier.

\section{Recherches au TIMA}
Le laboratoire est organisé en 5 équipes de recherche :
\begin{enumerate}
\item AMfoRS : Architectures and Methods for Resilient Systems
\begin{itemize}
	\item Spécification et vérification multi-niveaux d'architectures intégrant matériel et logiciel sur une puce : approches formelles et semi-formelles
	\item Modélisation, analyse et test au niveau système
	\item Sûreté de fonctionnement des systèmes intégrés : détection/tolérance de fautes, test en ligne, circuits auto-adaptatifs et auto-cicatrisants
	\item Evaluations de sûreté de fonctionnement : injection de fautes et alternatives (analytiques ou approches formelles), prédiction des effets du vieillissement sur la durée de vie
	\item Sécurité des systèmes intégrés : accélérateurs cryptographiques, contre-mesures contre les attaques matérielle
\end{itemize}
\item CDSI : Design of Integrated devices, Circuits and Systems
\begin{itemize}
	\item Circuits et systèmes asynchrones (IPs asynchrones, NoCs, GALS, logique asynchrone reconfigurable, etc.)
	\item Technologie asynchrone pour la sécurité matérielle
	\item Échantillonnage non uniforme et traitement du signal associé (capteurs, imageurs, circuits, algorithmes)
	\item Microsystèmes pour la médecine
	\item Micro générateurs pour microsystèmes autonomes
	\item Conception et technologies pour micro et nano systèmes intégrés
\end{itemize}
\item RIS : Robust Integrated Systems
\begin{itemize}
	\item Architectures tolérantes aux fautes et auto réglées
	\item Conceptions pour la Fiabilité face aux variabilités, le vieillissement, et les « soft errors »
	\item Gestion de la consommation multi-niveaux (du OS jusqu'au silicium)
	\item Auto test et auto réparation
	\item Algorithmes de routage tolérants aux fautes, auto-adaptatives, et à faible consommation
	\item Allocation et ordonnancement des taches au niveau réseau, tolérants aux fautes et avec prise en \item compte des variabilités, du vieillissement, et de la consommation
	\item Recouvrement des erreurs au niveau réseau
	\item Architectures mono-puces massivement parallèles robustes
	\item Architectures robustes pour les nanotechnologies
	\item Architectures durcies pour les applications spatiales
	\item Évaluation de robustesse et qualification: test sous radiation, injection des fautes
\end{itemize}
\item RMS : Reliable Mixed-signal Systems
\begin{itemize}
	\item Conception en vue du test de circuits analogiques, mixtes analogique-numérique et RF : nouvelles techniques de test intégré à bas coût pour les CANs, les circuits RF et imageur CMOS
	\item Estimation des métriques de test : faciliter l'évaluation et l'exploration des solutions de test intégré RF AMS.
	\item Calibrage des dispositifs RF/AMS : faire face à la chute du rendement de dispositifs RF ; calibrage à bas coût après le test de production
	\item Prédiction et contrôle de la qualité et de l'efficacité énergétique : application multimédia et appareils sans fil fonctionnant sur batterie
	\item Modélisation de haut niveau des systèmes hétérogènes et multi-physique : génération de modèles de haut niveau d'ordre réduit
\end{itemize}
\item SLS : System Level Synthesis
\begin{itemize}
	\item Architectures parallèles, configurables et reconfigurables
	\item Infrastructures logicielles pour les systèmes intégrés
	\item Synthèse, génération et simulation de systèmes numériques intégrés
\end{itemize}
\end{enumerate}

\section{Groupe SLS}
Les défis que posent l'intégration de systèmes complets en technologies CMOS nanométrique sont fortement liés au nombre phénoménal de processeurs que l'on peut implanter sur une puce. L'ITRS prévoit plus de 1000 processeurs en 2020 pour les applications destinées au grand public. Dans ce contexte, le groupe SLS a identifié les axes suivants comme étant majeurs :
\begin{itemize}
\item\ la définition d'architectures parallèles, configurables et reconfigurables; qui permettent d'exploiter pleinement les possibilitées offertes par ces technologies ;
\item\ la partie logicielle de ces systèmes; qui est de plus en plus importantes, la définition d'infrastructures permettant de déployer de manière efficace des applications à grande échelle sur des systèmes contraints en ressources est une piste majeure ;
\item\ l'intégration matérielle/logicielle; étant d'une grande complexité, des outils de synthèse, de génération et de simulation performant et \textit{scalable} sont nécessaires.
\end{itemize}