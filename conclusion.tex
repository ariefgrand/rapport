\chapter{Conclusion}

Ce stage avait pour but de réaliser un environnement qui permette de valider un mécanisme de
commutation de contexte sur FPGA au sein de laboratoire TIMA.
Au cours de ce stage de 22 semaines, j’ai pu réaliser avec succès les missions qui m’ont été confiées.
Travailler au sein de l’équipe SLS a été une expérience très enrichissante. 

J’ai pu, comme je le désirais en choisissant le sujet de stage, approfondir mes connaissances 
sur les plateformes reconfigurables de type FPGA, développer une application complète en matérielle et logicielle
et avoir une vision du projet d'ingénierie complet depuis l'idée jusqu'au produit.
De plus, j'ai pu réaliser cette expérience dans un laboratoire connu internationalement tel que TIMA, ce qui
m'a apporté une connaissance de l'environnement de travail de recherche et une expérience valorisante.
J'ai eu l'occasion de réaliser une solution à partir de mes idées pour répondre aux besoins de client.

La plus grande partie de ce travail a été de m'approprier la plateforme et le système d'exploitation utilisés.
C'était un challenge d'étudier et d'extraire des informations nécessaires et utiles des documentations techniques
d'un produit qui intègre beaucoup d'éléments comme la carte ZYBO.
La conception du démonstrateur a pris beaucoup de temps dans ce stage. Alors que le développement
des IP additionnelles en HDL ne donne pas beaucoup de difficultés comme j'ai eu assez d'expérience dans ce domaine, la conception
de l'architecture matérielle et l'interaction entre logiciel et matériel demandent des efforts considérables.

En faisant mon travail, j'ai réalisé que les idées de la conception ont un grand impact sur le travail
d'ingénierie. L'implémentation qui est effectuée par les ingénieurs ne sera pas optimale
sans supporter par une bonne conception auparavant.
Bien que la conception soit généralement liée avec la recherche,
un bon ingénieur doit être capable de développer la partie conception dans un projet,
ce qui est essentielle pour un ingénieur informatique industrielle et instrumentation.
Personnellement, je suis très chanceux d'avoir une occasion de faire un développement matériel et logiciel
à partir de la conception jusqu'au résultat.
Pour finir, j’ai énormément apprécié de travailler dans un environnement haute technologie et dans l’ambiance amicale propre au Laboratoire TIMA.