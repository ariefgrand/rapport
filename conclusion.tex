\chapter{Conclusion}

Ce stage avait pour but de réaliser un tel environnement qui permet de valider un mécanisme de
commutation de contexte sur un FPGA au sein de laboratoire TIMA.

Au cours de ce stage de 22 semaines, j’ai pu réaliser avec succès les missions qui m’ont été confiées.
Travailler au sein de l’équipe SLS a été une expérience très enrichissante. 

J’ai pu, comme je le désirais en choisissant le sujet de stage, approfondir mes connaissances 
en plateforme reconfigurable de type FPGA, développer une application complète en matérielle et logicielle
et aussi avoir une vision du projet d'ingénierie complet depuis l'idée jusqu'au produit.
De plus, j'ai pu réaliser cette expérience dans un laboratoire connu internationalement tel que TIMA, ce qui
m'a apporté une connaissance de l'environnement de travail au recherche et une expérience valorisante.
J'ai eu l'occasion de réaliser une solution à partir de mes idées pour répondre aux besoins de client,
qui est mon maître de stage.

La plus grande partie de ce travail a été de m'approprier à la plateforme et au système d'exploitation utilisés.
C'était un challenge d'étudier et d'extraire des informations nécessaires et utiles des documentations techniques
de la plateforme.

La conception du démonstrateur a été prise beaucoup de temps dans ce stage. Alors que le développement
des IPs additionnelles ne donne pas beaucoup de difficulté comme j'ai eu assez d'expérience dans ce domaine, la conception
de l'architecture matérielle et l'interaction entre logiciel et matériel demandent des efforts considérables.
En faisant mon travail, j'ai réalisé que les idées de la conception donnent un grand impact dans le travail
d'ingénierie, mais c'est le résultat qui est apprécie par le client. La grande valeur de la conception est
remarquable si le résultat généré est apparent.
C'est de la responsabilité de l'ingénieur de mettre en œuvre les conceptions aux solutions techniques.

Pour finir, j’ai énormément apprécié de travailler dans un environnement haute technologie et dans l’ambiance amicale propre au Laboratoire TIMA.