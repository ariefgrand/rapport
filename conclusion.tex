\chapter{Conclusion}

Ce stage avait pour but de réaliser un tel environnement qui permet de valider un mécanisme de
commutation de contexte sur un FPGA au sein de laboratoire TIMA.

Au cours de ce stage de 22 semaines, j’ai pu réaliser avec succès les missions qui m’ont été confiées.
Travailler au sein de l’équipe SLS a été une expérience très enrichissante. 

J’ai pu, comme je le désirais en choisissant le sujet de stage, approfondir mes connaissances 
en électronique numérique de liaison haut débit, et aussi avoir une vision de la conception des produits électronique de l’industrie.
De plus, j’ai pu réaliser ma première expérience professionnelle en France dans une entreprise internationale
telle que Digigram, ce qui m’a apporté une connaissance de l’environnement de travail et une expérience valorisante et inoubliable.
J’ai eu l’occasion de réaliser une pré-étude qui pourra permettre la conception de nouveaux produits. 

La plus grande partie de ce travail a été d’extraire les informations nécessaires et utiles pour Digigram 
à partir des documentations de normes. C’était un challenge de faire une synthèse de tous ces dossiers, 
que je n’ai pas pu maitriser techniquement profondément, dans une langue que je n’ai commencé à étudier que récemment.

En faisant mon travail, j’ai réalisé que les normes sont souvent trop idéales pour qu’il existe une implémentation pratique. 
Parfois des implémentations sont disponibles sur le marché sans être rattachées à une norme existante. 
C’est de la responsabilité de l’ingénieur de mettre en concordance la demande du client avec les solutions techniques disponibles.

Pour finir, j’ai énormément apprécié de travailler dans un environnement haute technologie et dans l’ambiance amicale propre au Laboratoire TIMA.