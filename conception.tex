\chapter{Conception}
\label{chap:concep}
\OnehalfSpacing

Le chapitre de conception va aborder les idées principales liées 
au système de validation de la commutation de contexte sur FPGA.
Comme expliqué dans l'introduction et la description du sujet de stage 
(cf. Section \ref{sec:sujet}), la méthode des \emph{checkpoints} 
est réalisée en ajoutant un programme à l'outil de HLS AUGH.
Durant la synthèse haut niveau, ce programme intègre une fonction
commutable dans le composant. Le mécanisme ajouté ne doit pas modifier le fonctionnement 
du composant de traitement ni diminuer sa performance de manière significative.

Un démonstrateur qui permet de lancer l'exécution et de réaliser la commutation
de contexte dans un traitement sera construit pendant le stage.
Le composant de traitement commutable sera programmé
et ses performances sont évaluées par le démonstrateur.
On adapte l'architecture générale de la plateforme de validation du composant de traitement
proposée dans le projet 3i5 dans le travail du stage car un flot d'exécution similaire est utilisé.

Dans le cadre du stage, un SoC programmable est utilisé comme la plateforme principale du démonstrateur.
Un poste de travail qui fonctionne comme un serveur avec un outil de génération de configuration
est utilisé pour générer le fichier de configuration de FPGA. 
Un stockage sur le réseau est mis en place afin de faciliter le transfert de données.
L'utilisateur peut accéder à ce démonstrateur en connectant son poste de travail au réseau.
L'architecture du démonstrateur de la commutation de contexte est visible sur la Figure \ref{fig:system}.
La conception de l'architecture matérielle et logicielle est expliquée dans les sections suivantes.

\begin{figure}[h]
	\centering
	\includegraphics[width=0.5\textwidth]{system}
	\caption{Le schéma du démonstrateur en général}
	\label{fig:system}
	\vspace{-2mm}
\end{figure} 

\section{Conception matérielle}
\label{sec:concephard}

Cette partie expose la partie matérielle du démonstrateur de la commutation de contexte.
La plateforme principale est un SoC programmable. Ce type de plateforme
combine un processeur et un FPGA dans la même puce avec un bus de communication interne. L'avantage d'une telle plateforme
est qu'elle est capable de reconfigurer la partie FPGA à partir du processeur, ce qui nous donne une plateforme plus autonome
par rapport aux autres familles de FPGA.

Le composant de traitement commutable, qui est testé par le démonstrateur, peut être programmée dans la partie 
FPGA grâce à l'outil de génération de configuration (cf. page précédente).
Cet outil transforme la description HDL du composant de traitement en un fichier de configuration (\emph{bitstream}), qui sera
chargé dans le FPGA. Le chargement du \emph{bitstream} sera effectué par le processeur du SoC.
L'outil de génération de configuration est installé sur un autre poste de travail en raison de la limitation de ressource du processeur embarqué.

La communication entre le processeur et le FPGA est effectuée à travers le bus de communication interne.
Il permet d'envoyer les données du processeur aux composants de traitements programmés dans le FPGA.
Ces composants de traitement doit intégrer une interface de communication
qui supporte le protocole de bus.

Afin de faciliter la validation du composant de traitement commutable, des composants 
annexes sont ajoutées à l'architecture matérielle du système.
Sont notamment ajoutées des composants annexes pour la gestion des entrées-sorties, la commutation de contexte et l'évaluation
des performances du système. Elles implémentent le protocole de communication pour que le processeur puisse
commander ou transférer les données.
Le schéma de l'architecture matérielle du démonstrateur est présenté
dans la Figure \ref{fig:concephard}.

\begin{figure}[h]
	\centering
	\includegraphics[width=0.8\textwidth]{concephard}
	\caption{Le schéma de l'architecture matérielle du démonstrateur}
	\label{fig:concephard}
	\vspace{-2mm}
\end{figure}

\section{Conception logicielle}
\label{sec:concepsoft}

Comme la plateforme utilisée est constitué d'un processeur, il est possible d'utiliser un système d'exploitation.
Le système d'exploitation nous permettra d'utiliser les drivers offerts selon la distribution choisie. 
Le système d'exploitation doit être léger et il doit permettre d'intégrer des bibliothèques de logiciels.
La conception de l'architecture logicielle repose sur la conception du système d'exploitation et les
bibliothèques de logiciels intégrés ainsi que tous ses drivers. 

Des bibliothèques de logiciels sont développées puis ajoutées au système d'exploitation afin de compléter les drivers offerts. 
Elles permettent l'accès à plusieurs drivers pour effectuer certaines fonctions.
Au niveau logiciel, plusieurs fonctions de base sont nécessaires pour réaliser le démonstrateur de la commutation
de contexte. Ces fonctions sont les suivants :
\begin{itemize}
	\item\ fonction d'entrée-sortie qui gère les entrées et les sorties de l'IP, 
	\item\ fonction context switch qui permet d'exécuter le mécanisme de la commutation de context, 
	\item\ fonction timer qui permet d'enregistrer le nombre de cycles,
	\item\ fonction post-processing qui calcule la performance de l'IP.
\end{itemize}
Les fonctions timer et post-processing sont importantes dans la réalisation de l'objectif du stage qui est l'évaluation des performances
du mécanisme proposé. Les bibliothèques de logiciels sont développées en langage C et intégrées dans le système d'exploitation.
Après l'intégration, l'utilisateur peut appeler les fonctions, soit avec un script bash, soit par une ligne de commande.

L'ensemble des fonctions ci-dessus est utilisé pour la validation de la commutation de contexte.
Afin de créer un flot d'exécution précis et répétable, une application principale est construite en script bash. 
Cette application est chargée du lancement du démonstrateur, à partir d'un composant de traitement commutable en RTL jusqu'à l'obtention du résultat de la validation.
L'utilisateur peut lancer toutes les exécutions à partir de cette application principale incluant la génération de fichier
\emph{bitstream} par l'outil.
Le schéma de l'architecture logicielle du démonstrateur est présenté dans la Figure \ref{fig:concepsoft}.

\begin{figure}[h]
	\centering
	\includegraphics[width=\textwidth]{concepsoft}
	\caption{Le schéma de l'architecture logicielle du démonstrateur}
	\label{fig:concepsoft}
	\vspace{-2mm}
\end{figure}

L'application, les bibliothèques de logiciels et les drivers sont intégrés dans
le système d'exploitation sur SoC. L'utilisateur avec son poste de travail peut accéder à
l'architecture logicielle à travers un canal \gls{SSH},
ce qui permet à l'utilisateur d'appeler l'application principale en script bash lors de la validation de mécanisme de \emph{context switch}.
Le canal de communication SSH est aussi utilisé pour faciliter la génération du fichier \emph{bitstream}.
Il permet à l'application principale de commander l'outil de génération de configuration installé dans un
autre poste de travail afin de générer le fichier \emph{bitstream}.

\section{Interaction entre l'architecture matérielle et logicielle}

Dans le cadre du stage, les architectures matérielle et logicielle sont
développées ensemble afin de réaliser un système complet,
sauf le composant de traitement commutable et des drivers qui sont déjà disponibles dans
le système d'exploitation.
L'interaction entre l'architecture matérielle et l'architecture logicielle est mise en place
pour que les deux architectures construisent un système propre.
Une illustration de l'interaction entre l'architecture matérielle 
et l'architecture logicielle du système est présentée dans la Figure \ref{fig:interaction}.

\begin{figure}[h]
	\centering
	\includegraphics[width=0.8\textwidth]{interaction}
	\caption{Interaction entre l'architecture matérielle et logicielle}
	\label{fig:interaction}
	\vspace{-2mm}
\end{figure}

L'architecture logicielle est capable d'envoyer des commandes ou
d'échanger des données avec l'architecture matérielle.
L'architecture logicielle du système est constitué d'une application, des bibliothèques de logiciels et des drivers
intégrés dans le système d'exploitation installé. Alors que l'architecture matérielle
est constitué du composant de traitement commutable et les composants annexes dans le FPGA.

Commandes et échanges de données sont pilotés par l'application. Cette application
contient le flot de la validation du mécanisme de \emph{context switch} implémenté.
Elle utilisera les bibliothèques de logiciels et les drivers afin d'exécuter les fonctions souhaitées.
Les drivers qui sont déjà présents dans le système d'exploitation permettront l'accès au
matériel.

D'autre part, toutes les IPs dans le FPGA sont intégrées dans un \emph{wrapper} grâce
à l'outil de génération de configuration. Ce \emph{wrapper} matériel est généré dans le projet de l'architecture matérielle
comme le composant \emph{top-level} de tous les autres composants.
Le \emph{wrapper} généré va permettre la transformation de l'ensemble des IP en un fichier \emph{bitstream} qui sera
chargé sur le FPGA.