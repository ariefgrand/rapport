\chapter{Conception}
\label{chap:concep}
\OnehalfSpacing

Ce chapitre parle des idées principales du système de la validation de commutation de contexte sur FPGA.
Comme expliqué dans la partie Introduction (cf. Chapitre \ref{chap:intro}) et la description du sujet de stage 
(cf. Section \ref{sec:sujet}), le mécanisme de la commutation de contexte développé par mon maître de stage
est intégré dans le logiciel de synthèse de haut niveau (synthèse logique).
 

\section{Architecture Matérielle}
\label{sec:concephard}

La conception de l'architecture matérielle repose sur le problème de connexion entre chaque plateforme
utilisée afin de construire un démonstrateur qui pourrait valider la commutation de contexte.
Dans le projet 3i5, 

L'architecture du système prévu est présente dans la Figure \ref{fig:system}

\begin{figure}[h]
	\label{fig:system}
	\centering
	\includegraphics[width=0.75\textwidth]{system}
	\caption{Schéma de l'architecture du système}
	\label{fig:system}
	\vspace{-2mm}
\end{figure}

\section{Architecture Logicielle}
\label{sec:concepsoft}

La conception de l'architecture logicielle est liée avec 

\section[Intéraction entre Les Deux]{Intéraction entre Les Architectures Matérielles et Logicielles}