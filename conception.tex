\chapter{Conception}
\label{chap:concep}
\OnehalfSpacing

Ce chapitre parle des idées principales liées au système de la validation de commutation de contexte sur FPGA.
Comme expliqué dans la partie Introduction (cf. Chapitre \ref{chap:intro}) et la description du sujet de stage 
(cf. Section \ref{sec:sujet}), le mécanisme de la commutation de contexte développé par mon maître de stage
est intégré dans le logiciel de synthèse logique AUGH. Durant la synthèse logique, un algorithme est ajouté
qui va permettre d'effectuer les étapes de la commutation de contexte. En tout cas, l'exécution de l'IP
doit donner le bon résultat à la fin comme il faut.

Un démonstrateur qui permet de lancer l'exécution de l'IP généré par AUGH et de réaliser la commutation
de contexte sera construit dans le stage.
On adapte l'architecture générale de la plateforme de la validation d'IP
proposé dans le projet 3i5\cite{Brisebard2015, Wicaksana2015} dans notre projet car ils utilisent
un flot d'exécution similaire. La plateforme est une plateforme SoC programmable conçue par Xilinx.
Un poste de travail qui fonctionne comme un serveur avec l'outil de backend Xilinx est ajouté pour générer le fichier de configuration. 
Un stockage de réseau est mis afin de faciliter le transfert de données.
L'utilisateur peut accéder ce démonstrateur en connectant sa station de travail au réseau.
L'architecture du démonstrateur de la commutation de contexte est présente dans la Figure \ref{fig:system}.

\begin{figure}[h]
	\centering
	\includegraphics[width=0.5\textwidth]{system}
	\caption{Le schéma du démonstrateur en général}
	\label{fig:system}
	\vspace{-2mm}
\end{figure} 

La conception de démonstrateur en général se compose de la conception en matériel et
de la conception en logiciel.

\section{Conception Matérielle}
\label{sec:concephard}

La conception matérielle du système décrit la partie matérielle du démonstrateur de la commutation de contexte.
Sa plateforme principale est une SoC programmable conçu par Xilinx. Ce type de plateforme
combine un processeur ARM et un FPGA dans la même puce avec un bus de communication interne. L'avantage de cette plateforme
est qu'elle est capable de reconfigurer la partie FPGA à partir du processeur ARM, ce qui nous donne une plateforme plus autonome
par rapport aux autres familles de FPGA.

L'IP généré par AUGH, qui sera testé par le démonstrateur, est programmé dans la partie FPGA grâce à l'outil de backend Xilinx.
Cet outil va transformer l'ensemble de l'IP en RTL à un fichier de configuration \emph{bitstream}, qui sera
configuré dans FPGA par le processeur. L'outil de backend Xilinx est installé à l'autre poste de travail en raison de la limitation de ressource.
Après la configuration, le processeur et le FPGA sont capables de se communiquer au travers de bus AXI. 
Le bus AXI est un bus de communication conçu par ARM pour les produits Xilinx. Afin de facilité la conversion de protocole
entre le bus AXI et l'IP, des interfaces sont ajoutées dans la partie FPGA. Ces interfaces sont compiler par l'outil de Xilinx
en même temps que l'IP.
Le schéma de l'architecture matérielle du démonstrateur expliquée est présenté
dans la Figure \ref{fig:concephard}.

\begin{figure}[h]
	\centering
	\includegraphics[width=0.6\textwidth]{concephard}
	\caption{Le schéma de l'architecture matérielle du démonstrateur}
	\label{fig:concephard}
	\vspace{-2mm}
\end{figure}

\section{Conception Logicielle}
\label{sec:concepsoft}

La conception de l'architecture logicielle repose sur la conception du système d'exploitation et tous ses
bibliothèques de logiciels intégrés ainsi que ses drivers. 
Comme la plateforme SoC programmable consitue un processeur ARM, il est possible d'installer un système d'exploitation.
Un système d'exploitation pour la plateforme nous permet d'utiliser les drivers offerts selon la distribution du système d'exploitation. Le système d'exploitation
utilisée est léger et il permet de l'intégration des bibliothèques de logiciels.

Les bibliothèques de logiciels sont ajoutées au système d'exploitation afin de compléter les drivers qui sont déjà offerts par
le système d'exploitation. Elles permettent d'accès aux plusieurs drivers pour effectuer certaines fonctions.
Au niveau de logiciel, il y a plusieurs fonctions de base nécessaires pour réaliser le démonstrateur de la commutation
de contexte. Ces fonctions sont les suivants :
\begin{itemize}
	\item\ fonction d'entrée-sortie qui gère les entrées et les sorties de l'IP, 
	\item\ fonction context switch qui permet d'exécuter le mécanisme de la commutation de context, 
	\item\ fonction timer qui permet d'enregistrer le nombre de cycles,
	\item\ fonction post-processing qui calcule la performance de l'IP.
\end{itemize}
Les fonctions timer et post-processing sont importantes dans la réalisation de l'objectif de stage qui est l'évaluation des performances
du mécanisme proposé. Les bibliothèques de logiciels sont développées en langage C et intégrées dans le système d'exploitation.
Après l'intégration, l'utilisateur peut appeler les fonctions, soit dans le script de bash, soit par une ligne de commande.

L'ensemble des fonctions ci-dessus est utilisé pour lancer la validation de la commutation de contexte.
Afin de créer un flot d'exécution de manière précis et répétable, une application principale est construite en script de bash. 
Cette application sera chargée de lancement du démonstrateur, à partir d'un IP en RTL jusqu'au résultat de la validation.
L'utilisateur peut lancer toutes les exécutions à partir de cette application principale incluant la génération de fichier
\emph{bitstream} dans l'outil de backend Xilinx.
Le schéma de l'architecture logicielle du démonstrateur expliquée est présenté dans la Figure \ref{fig:concepsoft}.

\begin{figure}[h]
	\centering
	\includegraphics[width=0.7\textwidth]{concepsoft}
	\caption{Le schéma de l'architecture logicielle du démonstrateur}
	\label{fig:concepsoft}
	\vspace{-2mm}
\end{figure}

\section{Interaction entre L'Architecture Matérielle et Logicielle}

L'interaction entre la partie matérielle et la partie logicielle est présentée dans la
Figure \ref{fig:interaction}. Cette interaction est réalisée sous la forme des commandes et des données.
Les commandes sont données par l'application au matériel pour réaliser tout le fonctionnement.
La partie matérielle constitue l'IP généré par AUGH et les interfaces qui sont intégrés
dans un wrapper hardware alors que la partie logicielle constitue
l'application, les bibliothèques de logiciels et les drivers dans un système d'exploitation. 
Le wrapper hardware est généré par l'outil de backend Xilinx, programmé dans la partie FPGA.

\begin{figure}[h]
	\centering
	\includegraphics[width=0.35\textwidth]{interaction}
	\caption{Interaction entre l'architecture matérielle et logicielle}
	\label{fig:interaction}
	\vspace{-2mm}
\end{figure}