\chapter{Les principes de base technologiques}
\label{chap:principe}
\OnehalfSpacing
\section{\emph{Field-Programmable Gate Array (FPGA)}}
\label{sec:fpga}
Les FPGA (Field-Programmable Gate Arrays), traduit en français par circuits prédiffusés 
programmables\cite{hubinweb},
sont des composants \gls{VLSI}  
entièrement reconfigurables. Contrairement aux circuits prédiffusés conventionnels, les circuits
prédiffusés programmables ne demandent pas de fabrication spéciale en usine, ni de systèmes de 
développement coûteux. Inventés par Ross Freeman, le co-fondateur de la société Xilinx en 1985, les FPGA, dans la famille
des \glspl{ASIC}, se situent entre les réseaux logiques programmables et les prédiffusés. 

Les circuits FPGA sont constitués d'une matrice de blocs logiques configurables permettant de réaliser des fonctions
combinatoires et des fonctions séquentielles. Tout autour de ces blocs logiques, on trouve des blocs entrées/sorties
dont le rôle est de gérer les entrées-sorties réalisant l'interface avec les modules extérieurs.
L'ensemble est relié par un réseau d'interconnexions programmable, présenté dans la Figure \ref{fig:layoutfpga}. 

Un bloc logique est de manière générale constitué d'une table de correspondance
(\emph{Look-Up-Table} ou \gls{LUT}) et d'une bascule (\emph{Flip-Flop}). 
La \gls{LUT} sert à implémenter
des équations logiques ayant généralement 4 à 6 entrées et une sortie. 
Elle peut toutefois être utilisée comme une petite
mémoire, un multiplexeur ou un registre à décalage. 
Le registre permet de mémoriser un état (machine séquentielle) ou
de synchroniser un signal (pipeline).
La configuration du circuit est mémorisée dans une mémoire \gls{SRAM} 
à chaque mise sous tension. Comme la programmation est dynamique,
cette configuration doit être stockée dans un élément externe (par exemple une mémoire \gls{ROM} ou carte \gls{SD}).

\begin{figure}[h]
	\centering
	\includegraphics[width=0.3\textwidth]{layoutfpga}
	\caption{Diagramme d'un FPGA\cite{fpgaprototype}}
	\label{fig:layoutfpga}
	\vspace{-2mm}
\end{figure}

La configuration du circuit dans FPGA est possible grâce à un fichier de configuration, connu sous le nom
\emph{bitstream}. Un fichier \emph{bitstream} est un fichier binaire qui contient les données de configuration
de FPGA. Ce fichier est chargé dans le FPGA afin de programmer le circuit souhaité.
Le \emph{bitstream} est généré à partir d'une description fonctionnelle de circuit.
Cette description peut être écrite en \emph{Hardware Description Language} ou HDL, normalement
en langage Verilog ou VHDL. Le flot de programmation des FPGAs à partir d'un code en HDL est présenté dans la Figure \ref{fig:fpgaflow}. 

\begin{figure}[h]
	\centering
	\includegraphics[width=0.4\textwidth]{fpgaflow_hauck}
	\caption{Le flot de programmation typique des FPGA}
	\label{fig:fpgaflow}
	\vspace{-2mm}
\end{figure}

La synthèse logique est une étape qui consiste à compiler la description fonctionnelle d'un circuit à l'aide d'un outil de synthèse et
d'une bibliothèque de cellules logiques pour générer en sortie une description en portes logiques (\emph{gate-level netlist}).
L'étape de «\emph{Technology Mapping}» va associer la \emph{netlist} générée aux ressources spécifiques: LUT, bascule, BRAM, etc.
Ces ressources sont mises en place et interconnectées sur le FPGA à l'étape Placement et Routage (\emph{Routing}).
Enfin, le fichier de configuration est généré à partir des informations obtenues aux étapes précédentes.

Il existe un flot de programmation des FPGAs avec un niveau plus abstrait. Ce flot est effectué à partir de code ANSI C/C++/SystemC
au lieu de description en HDL, qui sera expliqué dans la Section \ref{sec:hls}.

Les FPGA sont utilisés dans diverses applications nécessitant de l'électronique numérique (télécommunications, aéronautique, transports...).
Ils sont également utilisés pour le prototypage d'ASIC.
Ils sont généralement plus lents et consomment davantage d'énergie 
que leur équivalent en ASIC (Application Specific Integrated Circuit). Cependant, ils ont plusieurs avantages :
\begin{itemize}
    \item\ délai de mise sur le marché plus court, car ce sont des composants standards,
    \item\ temps de développement plus court, car on réutilise des fonctions de base et 
	    la reconfigurabilité autorise une validation préalable moins stricte,
    \item\ coût inférieur pour de petites séries (moins de 10 000 unités). 
	Avec l'évolution technologique, cette quantité tend à augmenter : en effet, le prix d'une puce est 
	proportionnel à sa surface, qui diminue avec la finesse de gravure, tandis que les coûts initiaux 
	pour fabriquer un ASIC (conception, tests, masques de gravure) sont en forte augmentation.
\end{itemize}

Plusieurs FPGA modernes possèdent la possibilité d'être reconfigurés 
(on parle de configuration lorsqu'il s'agit de programmation du matériel) partiellement à la volée. 
Ceci permet d'obtenir des systèmes reconfigurables - par exemple un processeur dont 
les instructions changent dynamiquement en fonction des besoins.

Les FPGA modernes sont assez vastes et contiennent suffisamment de mémoire pour être configurés 
et héberger un cœur de processeur ou un DSP, afin d'exécuter un programme. 
On parle dans ce cas de processeur soft-core, par opposition aux microprocesseurs hard-core enfouis dans le silicium. 
Les fabricants de FPGA intègrent même un ou plusieurs cœurs de processeur hard-core 
sur un même composant afin de conserver les ressources logiques configurables du composant. 
Ceci n'exclut pas l'utilisation de processeur soft-core possédant de nombreux avantages. 
On tend donc vers des Systems On Chip (\gls{SoC}), comme pour le microcontrôleur il y a quelques décennies, 
avec en plus de la logique configurable selon l'utilisateur. Une des familles FPGAs SoC
qui intègrent un processeur hard-core et soft-core dans le même puce est Zynq-7000\cite{ug585} conçu par Xilinx.

\section{Synthèse haut niveau ou \emph{High Level Synthesis} (HLS)}
\label{sec:hls}
La synthèse haut niveau est un processus qui interprète une spécification au niveau algorithmique (un programme) en une architecture
au niveau transferts de registres (\gls{RTL}) pour la réalisation d'un circuit\cite{Vijay1996}.
Il est également appelé \emph{behavioral synthesis} ou \emph{algorithmic synthesis}.

Le but de la synthèse haut niveau est d'améliorer la productivité de la conception 
de composant de traitement. En effet, la productivité ne suit pas la capacité d'intégration. Cette capacité d'intégration
respecte la loi de Moore qui avait prédit un doublement de la capacité de calcul chaque 18 mois.
La synthèse haut niveau a plusieurs avantages :
\begin{itemize}
	\item\ réduction du temps de conception,
	\item\ meilleure exploration architecturale,
	\item\ élévation du niveau d'abstraction: code plus portable, spécification de taille réduite (moins d'erreurs) et simulation plus rapide.
\end{itemize}

L'outil de HLS transforme une spécification haut niveau à une implémentation. Il génère une architecture
afin de réaliser la spécification efficacement. En plus des interfaces de communication et des mémoires,
l'architecture générée est décrite au niveau RTL et elle est constituée d'un chemin de données ou partie opérative (registres, multiplexeurs,
unités fonctionnelles et bus) et d'une partie contrôle selon les besoins et les contraintes. Un schéma représente
l'architecture générée par l'outil de HLS se trouve dans la Figure \ref{fig:pcpo}.

\begin{figure}[h]
	\centering
	\includegraphics[width=0.65\textwidth]{pcpo}
	\caption{Schéma PC-PO}
	\label{fig:pcpo}
	\vspace{-2mm}
\end{figure}

Les tâches exécutées par un outil de synthèse haut niveau à partir d'une description haut niveau,
d'une bibliothèque d'élements RTL et des contraintes sont les suivantes :
(cf. Figure \ref{fig:hlsflow})

\begin{enumerate}
	\item
	
	compilation de la spécification,
	\item
	
	allocation des ressources matérielles (unités fonctionnelles, éléments de stockages, bus, etc),
	\item
	
	ordonnancement des opérations à cycle d'horloge,
	\item
	
	assignation sur les ressources allouées,
	 \item
	
	génération de l'architecture RTL.
\end{enumerate}

\begin{figure}[h]
	\centering
	\includegraphics[width=0.65\textwidth]{hlsflow}
	\caption{Les étapes de synthèse haut niveau\cite{Coussy2009}}
	\label{fig:hlsflow}
	\vspace{-2mm}
\end{figure}

La synthèse haut niveau est commencée par la compilation de la spécification fonctionnelle.
Cette étape transforme l'entrée à une représentation intermédiaire. Ce modèle produit par la compilation
expose les dépendances des données et des contrôles entre les opérations. Les dépendances des données et des contrôles sont représentées
par un graphe de flux des contrôles et données (CDFG).

L'allocation définit le type et le nombre des ressources matérielles requises pour répondre aux contraintes de conception.
Ces éléments sont pris de la bibliothèque d'éléments RTL.
En fonction de l'outil de HLS, plusieurs éléments sont ajoutés pendant l'ordonnancement et l'assignation.
Par exemple, les éléments de connectivité (bus ou connexion point à point) peuvent être ajoutés avant ou
après l'assignation et l'ordonnancement de tâches.

Toutes les opérations requises dans la spécification doivent être ordonnancées en cycles.
En fonction de l'élément fonctionnel où l'opération est définie, elle peut être ordonnancée
en un cycle ou plusieurs cycles d'horloge. L'ordonnancement est normalement implémenté avec deux types de contraintes :
l'ordonnancement avec les contraintes de ressources et l'ordonnancement avec les contraintes de temps.

Toutes les variables avec valeur doivent être assignées à une unité de stockage. Par contre, plusieurs variables
avec une durée de vie exclusive ou non-chevauchant peuvent être assignés aux même unités de stockage.
Chaque opération dans la spécification doit être assignée à une des unités fonctionnelles capables de l'exécuter.
L'optimisation est faite s'il y a plusieurs unités avec la capacité similaire.
L'assignation de stockage et d'unités fonctionnelles dépend aussi de l'assignation de connexion (réalisé
par les unités de connexion comme bus ou multiplexeur).

Les différentes étapes précédentes constituent la synthèse haut niveau. L'architecture générée est constituée de deux éléments principaux : une partie contrôle
et une partie opérative. La partie contrôle est modélisée par une automate d'états finis (FSM). La partie opérative est
l'interconnexion des éléments comme les unité fonctionnelles, éléments de stockage, etc.

\subsubsection*{L'outil de HLS AUGH}

AUGH est un outil de synthèse haut niveau conçu pour la génération automatique d'accélérateur matériel
pour FPGA, sous des contraintes de ressources\cite{Prost2014}. Développé par 
Adrien Prost-Boucle, un chercheur postdoctoral au laboratoire TIMA, 
AUGH est capable de générer une description générique VHDL à partir
d'une application écrite en langage C. 

L'outil de HLS AUGH est développé comme un outil libre et gratuit (\emph{free and open-source}).
C'est un outil lancé par une ligne de commande et conçu pour un système Linux. 
Les avantages de l'outil de HLS AUGH sont :
\begin{itemize}
	\item
	
	l'exploration de l'espace de conception automatique et rapide,
	
	\item
	
	respect strict des contraintes de ressources (nombre de LUT, bascule, BRAM, etc),
	
	\item 
	
	respect des contraintes de fréquence,
	
	\item 
	
	adapté aux utilisateurs non experts en conception de circuit numérique,
	
	\item
	
	possibilité d'extension avec des greffons (technologie des FPGAs supplémentaire, carte électronique, interfaces de communication, etc).

\end{itemize}

\section{Commutation de contexte (\emph{Context-Switch}) sur FPGA}
\label{sec:contextswitch}
En informatique, la commutation de contexte ou \emph{context-switch} 
consiste à sauvegarder l'état d'un processus pour restaurer à la place celui d'un autre
dans le cadre d'ordonnancement d'un système d'exploitation multitâche.
Dans les applications sur les FPGAs moderne, le mécanisme similaire a été envisagé. La commutation de contexte sur FPGA
permet d'arrêter une tâche programmée dans le FPGA et de continuer une autre tâche à la volée.

Le principe de la commutation de contexte repose sur la récupération et la restauration de contexte de la tâche configurée. 
Le contexte correspond aux données dans les registres et le compteur de programme à un moment donné.
Sur un FPGA, la commutation de contexte est effectuée entièrement en matériel. L'état du circuit est enregistré
sous la forme d'un contexte et sera restauré lorsque la tâche est continuée. 
Avant la restauration du contexte, le FPGA est reconfiguré avec la tâche correspondant au contexte car le contexte
ne contient pas les données de configuration.

Dans le cadre de sa thèse, M. Alban Bourge a développé un mécanisme qui permet 
la récupération de contexte aux points de contrôles ou \emph{checkpoints}.
Un point de contrôle est défini comme un état d'une tâche où la commutation de contexte est 
autorisée\cite{Bourge2015}. Le concept de \emph{checkpoint} est utilisé principalement 
dans les applications tolérantes aux erreurs. Le contexte
de l'application est sauvegardé à chaque fois qu'un \emph{checkpoint} est atteint. 
Lorsqu'il y a une erreur, le système est capable de restaurer
l'état de l'application au \emph{checkpoint} sauvegardé.
Du même principe, la tâche est reprise à partir des \emph{checkpoints} 
après la restauration du contexte.

Un exemple de scénario d'une commutation de contexte sur FPGA est illustré dans la Figure \ref{fig:switch}.
Une tâche (tâche 1) est configurée et exécutée dans un FPGA. Au milieu de l'exécution, la tâche est interrompue par
une demande de l'arrêt de tâche (demande de préemption).
Le mécanisme proposé ne permet pas un arrêt de la tâche immédiatement. Par contre, elle continue l'exécution pendant
le temps $t_c$ jusqu'à ce qu'un \emph{checkpoint} soit atteint. Lorsque l'exécution est arrivée au \emph{checkpoint},
l'état du circuit est enregistré en récupérant le contexte. Après une reconfiguration du FPGA, le traitement de l'autre tâche (tâche 2)
est effectué. Cette tâche est arrêté de la même manière que la tâche 1. Lorsque la tâche est reprise, le FPGA
est reprogrammé avec le \emph{bitstream} de tâche 1. La restauration du contexte permet de reprendre l'exécution
de la tâche 1 là où elle avait été arrêtée.

\begin{figure}[h]
	\centering
	\includegraphics[width=0.9\textwidth]{context}
	\caption{Exemple de commutation de contexte matériel entre deux applications\cite{Bourge2015}}
	\label{fig:switch}
	\vspace{-2mm}
\end{figure}

\section{Détermination des \emph{checkpoints}}

Comme déjà expliqué dans la section précédente, la récupération
de contexte (et sa restauration) n'est autorisée qu'aux certains points pendant l'exécution, ce qu'on
appelle les \emph{checkpoints}. Du matériel est ajouté au composant de traitement
pour que la récupération et la restauration de contexte soient possible (l'ajout de fonction
commutation) aux \emph{checkpoints}. Au lieu de réaliser le composant de traitement commutable pendant la synthèse logique et
le placement-routage avec un outil du fabricant de FPGA dont l'accès est très limité, 
la solution proposée est de faire le composant durant l'étape de synthèse haut niveau\cite{Bourge2015}.

Un programme a été ajouté à l'outil de HLS AUGH pour la détermination des \emph{checkpoints}.
Le programme choisit les \emph{checkpoints} qui donnent la meilleure efficacité au niveau de ressource
et de la taille de données et qui correspondent aux contraintes données 
(temps de latence maximum, largeur de données, ...) par l'utilisateur.
Le matériel qui permet d'effectuer la commutation de contexte est ajouté aux \emph{checkpoints}
par l'outil de HLS AUGH.
Ce mécanisme de la détermination des \emph{checkpoints} est indépendant de la technologie utilisée car elle
est faite durant l'étape de synthèse haut niveau.
