\chapter{Les Principes de Base Technologiques}
\label{chap:principe}
\OnehalfSpacing
\section{\emph{Field Programmable Gate Array (FPGA)}}
\label{sec:fpga}
 
Les FPGA ( Field Programmable Gate Arrays ou "réseaux logiques programmables" ) sont des composants VLSI 
entièrement reconfigurables ce qui permet de les reprogrammer à volonté afin d'accélérer notablement certaines 
phases de calculs\cite{fpgaweb}.

Les circuits FPGA sont constitués d'une matrice de blocs logiques programmables entourés de blocs d'entrée 
sortie programmable. L'ensemble est relié par un réseau d'interconnexions programmable. 

\begin{figure}[h]
	\label{fig:layoutfpga}
	\centering
	\includegraphics[width=0.5\textwidth]{layoutfpga}
	\caption{Diagramme d'un FPGA\cite{fpgaprototype}}
	\vspace{-2mm}
\end{figure}

\begin{figure}[h]
	\label{fig:fpgaflow}
	\centering
	\includegraphics[width=0.5\textwidth]{fpgaflow_hauck}
	\caption{Le flux de dessin sur FPGA}
	\vspace{-2mm}
\end{figure}

\section{Outils de synthèse}

La synthèse logique est une étape qui consiste à compiler la description fonctionnelle d'un circuit 
à l'aide d'un outil de synthèse et d'une bibliothèque de cellules logiques. 
Cette description peut être écrite en langage Verilog ou VHDL et ne doit pas comporter d'éléments 
comportementaux non compréhensibles par l'outil de synthèse.

\section{Commutation de contexte sur FPGA}
\label{sec:contextswitch}

\Blindtext[1]

\begin{figure}[h]
	\label{fig:switch}
	\centering
	\includegraphics[width=\textwidth]{switch}
	\caption{Exemple de changement de contexte matériel sur deux FPGA}
	\vspace{-2mm}
\end{figure}
